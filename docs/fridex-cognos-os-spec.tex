\documentclass[11pt, a4paper]{report}

% =============================================================================
% Fridex / FridayX Canonical Spec (Live State)
%
% - Single document that merges the "manifesto" + "mathematical spec".
% - Aligned with Friday app's live architecture (Tauri Rust + Codex app-server).
% - Designed to compile with XeLaTeX (Tectonic) without relying on system fonts.
% =============================================================================

% --- PAGE / LANGUAGE ---
\usepackage[a4paper, top=2.5cm, bottom=2.5cm, left=2cm, right=2cm]{geometry}
\usepackage{fontspec}
% NOTE: We compile with XeLaTeX/Tectonic (XeTeX). Babel's `bidi=basic` requires LuaLaTeX.
% Turkish is LTR, so we keep bidi off for engine compatibility.
\usepackage[main=turkish, english]{babel}

% Avoid system-font dependencies; TeX Gyre ships with TeX distributions.
\setmainfont{TeX Gyre Heros}
\setsansfont{TeX Gyre Heros}
\setmonofont{TeX Gyre Cursor}

% --- MATH / BOXES / LAYOUT ---
\usepackage{amsmath}
\usepackage{amssymb}
\usepackage{mathtools}
\usepackage{xcolor}
\usepackage{enumitem}
\usepackage{tcolorbox}
\usepackage{booktabs}
\usepackage{titlesec}
\usepackage{fancyhdr}
\usepackage{setspace}
\usepackage{multicol}
\usepackage{hyperref}
\hypersetup{
  colorlinks=true,
  linkcolor=black,
  urlcolor=black,
  citecolor=black
}

\setlist[itemize]{label=-, leftmargin=1.2em}
\onehalfspacing

% --- BRAND PALETTE ---
\definecolor{fridex_cyan}{RGB}{0, 191, 255}
\definecolor{fridex_navy}{RGB}{10, 20, 30}
\definecolor{math_bg}{RGB}{245, 252, 255}

% --- TITLES ---
\titleformat{\chapter}[display]
  {\normalfont\bfseries\color{fridex_cyan}}
  {\filleft\Huge\thechapter}
  {1ex}
  {\titlerule\vspace{1ex}\filleft\Huge}

\titleformat{\section}{\color{fridex_cyan}\Large\bfseries}{\thesection}{1em}{}[\titlerule]
\titleformat{\subsection}{\color{fridex_cyan!70!black}\large\bfseries}{\thesubsection}{1em}{}

% --- BOXES ---
\newtcolorbox{fridexbox}[2][]{
  colback=white,
  colframe=fridex_cyan,
  fonttitle=\bfseries,
  title=#2,
  arc=3mm,
  boxrule=1pt,
  #1
}

\newtcolorbox{mathlogic}[1]{
  colback=math_bg,
  colframe=fridex_cyan!50!black,
  boxrule=0.5pt,
  arc=1mm,
  title=\textit{Teknik Formülasyon: #1},
  fonttitle=\small\bfseries
}

% --- HEADER / FOOTER ---
\pagestyle{fancy}
\fancyhf{}
\lhead{\textcolor{fridex_cyan}{Fridex / FridayX Kanonik Spesifikasyon}}
\rhead{\thepage}
\lfoot{Bekircan Akyüz}
\rfoot{Sürüm 4.0 (Canlı Durum)}

\begin{document}

% =============================================================================
% COVER
% =============================================================================
\begin{titlepage}
  \centering
  \vspace*{3cm}
  {\Huge \textbf{\textcolor{fridex_cyan}{FRIDEX}}} \\
  \vspace{0.5cm}
  {\Large \textbf{Otonom Bilişsel İşletim Sistemi (Cognos-OS)}} \\
  \vspace{0.35cm}
  {\large \textbf{FridayX Editör + Ajan Orkestrasyonu ile Kanonik Teknik Spesifikasyon}} \\
  \vspace{2cm}
  \begin{tcolorbox}[colback=fridex_navy, colframe=fridex_navy, arc=0mm]
    \centering \textcolor{white}{"Zeka, verideki düzeni değil, belirsizlikteki fırsatı görmektir."}
  \end{tcolorbox}
  \vfill
  \textbf{Yazar:} Bekircan Akyüz \\
  \textbf{Rol:} Baş Sistem Mimarı \\
  \textbf{Tarih:} \today \\
  \vspace{1cm}
\end{titlepage}

\tableofcontents
\newpage

% =============================================================================
% PART I: VISION
% =============================================================================
\chapter{Stratejik Vizyon ve Sistem Felsefesi}

\section{Giriş: Fridex ve FridayX Nedir?}
\textbf{Fridex}, kullanıcı ile simbiyotik bağ kuran bir \textbf{Bilişsel İşletim Sistemi (Cognos-OS)} vizyonudur.
\textbf{FridayX} ise bu vizyonu üretime taşıyan, yerel çalışma alanlarında (workspaces) \textbf{Codex ajanlarını} orkestre eden macOS odaklı bir editör ve kontrol düzlemidir.

Bu döküman iki hedefi aynı anda taşır:
\begin{itemize}
  \item \textbf{Vizyon:} Sistem felsefesi, tasarım motivasyonları, kullanıcı niyeti ve belirsizlikle baş etme.
  \item \textbf{Spesifikasyon:} Uygulanabilir mimari sözleşmeler, matematiksel modeller, test ve doğrulama süreci.
\end{itemize}

\section{Temel İlkeler}
Fridex'in gelişimi üç ana sütun üzerine inşa edilir:
\begin{itemize}
  \item \textbf{Otonom Grounding (Topraklama):} Dijital sinyali fiziksel/sistem gerçekliğiyle eşleştirme.
  \item \textbf{Privacy-by-Architecture (Mimariden Gelen Gizlilik):} Verinin mümkün olduğunca yerelde (local-first) işlenmesi.
  \item \textbf{Epistemik Merak:} Sistemin bilgi eksikliklerini aktif biçimde kapatmaya çalışması.
\end{itemize}

\section{Üretim Guardrail'leri (Özet)}
FridayX üretim rehberi; \textbf{DRY, KISS, YAGNI} ile birlikte performans ve güvenliği “birinci sınıf gereksinim” kabul eder:
\begin{itemize}
  \item \textbf{Multi-tenant izolasyonu:} \texttt{tenant\_id} bağlamı zorunlu; çapraz tenant erişim \textbf{yasak}.
  \item \textbf{RBAC/ABAC:} deny-default; kararlar audit edilebilir olmalı.
  \item \textbf{Observability:} log/metric/trace; PII maskeleme; \texttt{tenant\_id} etiketleme.
  \item \textbf{Performans:} P95 $< 100ms$ hedefi; hot-path'lerde algoritmik verimlilik.
\end{itemize}

\newpage

% =============================================================================
% PART II: LIVE ARCHITECTURE
% =============================================================================
\chapter{Canlı Mimari: FridayX Yığını ve Veri Akışı}

\section{Hibrit Yazılım Yığını}
FridayX, bir masaüstü uygulaması olarak aşağıdaki canlı mimariyle çalışır:
\begin{itemize}
  \item \textbf{Frontend:} React + Vite (UI, editör, önizlemeler, olay abonelikleri).
  \item \textbf{Backend (App):} Tauri Rust (platform köprüleri, workspace yönetimi, dosya IO, ajan orkestrasyonu).
  \item \textbf{Backend (Ajan Katmanı):} Workspace başına \texttt{codex app-server} süreçleri; JSON-RPC üzerinden event akışı.
\end{itemize}

\begin{mathlogic}{İletişim Modeli (JSON-RPC)}
\[
\text{UI} \xleftrightarrow[\text{invoke/events}]{\text{Tauri IPC}}
\text{Rust} \xleftrightarrow[\text{stdin/stdout}]{\text{JSON-RPC}}
\text{codex app-server}
\]
\end{mathlogic}

\section{IPC ve Veri Akışı (Pratik)}
FridayX tarafında kritik sözleşme: \textbf{initialize} \textrightarrow{} \textbf{initialized} olmadan request gönderilmez.
Threads listelenir (\texttt{thread/list}), devam ettirilir (\texttt{thread/resume}), arşivlenir (\texttt{thread/archive}).

\section{Editör ve Önizleme İlkesi}
Editör \textbf{kod görünümü} ile \textbf{önizleme}yi ayrıştırır:
\begin{itemize}
  \item Markdown dosyaları için \textbf{HTML preview}.
  \item LaTeX dosyaları için \textbf{PDF compile + preview}.
\end{itemize}
Bu döküman, FridayX editörde LaTeX önizleme hattının “tek dosya” modunda kusursuz çalışmasını hedefler.

\newpage

% =============================================================================
% PART III: COGNITIVE ENGINE (ACTIVE INFERENCE)
% =============================================================================
\chapter{Bilişsel Motor ve Aktif Çıkarım}

\section{Aktif Çıkarım (Active Inference) Teorisi}
Fridex, belirsizliği (entropy) azaltmaya çalışan bir ajan olarak modellenir.
Bu yaklaşım, Karl Friston'ın “Serbest Enerji Prensibi” çizgisinde özetlenir.

\subsection{Varyasyonel Serbest Enerji (VFE)}
Sistem, duyusal girdi ile içsel dünya modeli arasındaki “sürprizi” minimize eder:
\begin{equation}
  \mathcal{F} = \mathrm{D}_{KL}[Q(s) \,\|\, P(s,o)]
\end{equation}
Burada $Q(s)$ sistemin inancı, $P(s,o)$ ise dünya durumudur.

\subsection{Merak Mekanizması (EFE)}
Fridex sadece komutları yerine getirmez; gelecekteki belirsizliği azaltmak için epistemik bilgi toplar:
\begin{mathlogic}{Beklenen Serbest Enerji}
  \[
  G(\pi) \approx \text{Goal Achievement} + \text{Information Gain}
  \]
\end{mathlogic}

\section{Bayesyen Niyet Çözümleme}
\[
P(\text{Niyet} \mid \text{Prompt, Sistem Durumu}) =
\frac{P(\text{Prompt} \mid \text{Niyet})\, P(\text{Niyet})}{P(\text{Prompt})}
\]

\newpage

% =============================================================================
% PART IV: KNOWLEDGE GRAPH + MEMORY
% =============================================================================
\chapter{Hiyerarşik Hafıza: Çizge Tabanlı Model ve Skorlama}

\section{Çizge Tabanlı Hafıza Topolojisi}
Bilgiler; birbirinden kopuk vektörler yerine ilişkili düğümler olarak ele alınır.

\begin{fridexbox}{Tanım: Hafıza Uzayı}
Hafıza uzayı $\mathcal{M}$, bir çizge $G$ ile modellenir:
\[
G = (V, E, \mathcal{A}, \mathcal{R})
\]
\begin{itemize}
  \item $V$: varlıklar kümesi (User, File, Project, Task).
  \item $E \subseteq V \times V$: varlık ilişkileri.
  \item $\mathcal{A}$: düğüm öznitelik vektörleri (embeddings).
  \item $\mathcal{R}$: ilişki tipleri (örn. \texttt{OWNS}, \texttt{DEPENDS\_ON}).
\end{itemize}
\end{fridexbox}

\subsection{Hibrit Erişim Skorlaması}
Bir sorgu $q$ ile düğüm $v$ arasındaki alaka düzeyi:
\[
S(q, v) =
\alpha \cdot
\frac{\mathbf{e}_q \cdot \mathbf{e}_v}{\|\mathbf{e}_q\| \|\mathbf{e}_v\|}
 + (1-\alpha)\cdot e^{-\lambda \cdot d(v, v_{focus})}
\]
\begin{itemize}
  \item $\mathbf{e}_q, \mathbf{e}_v$: gömme vektörleri.
  \item $d(\cdot)$: en kısa yol uzaklığı.
  \item $\lambda$: sönümleme katsayısı.
\end{itemize}

\newpage

% =============================================================================
% PART V: POMDP BELIEF UPDATE
% =============================================================================
\chapter{Durum Tahmini: POMDP ve İnanç Güncelleme}

\section{İnanç Durumu (Belief State)}
\[
B_t(s) = P(S_t = s \mid O_{1:t}, A_{1:t-1})
\]
\begin{itemize}
  \item $O_t$: gözlem (kullanıcı girdisi, dosya içeriği, sistem telemetrisi).
  \item $A_t$: eylem (dosya yazma, yanıt üretme, süreç başlatma).
\end{itemize}

\section{Bayesyen Güncelleme}
\[
B_{t+1}(s') = \eta \cdot P(O_{t+1} \mid s') \sum_{s} P(s' \mid s, A_t) B_t(s)
\]

\newpage

% =============================================================================
% PART VI: DECISION - EXPECTED FREE ENERGY
% =============================================================================
\chapter{Karar Mekanizması: Beklenen Serbest Enerji ve Fayda}

\section{Beklenen Serbest Enerji (EFE)}
\[
G(\pi) = \sum_{\tau} G(\pi, \tau)
\]

\[
G(\pi, t) \approx
 - \mathbb{E}_{Q}[\ln P(O_t \mid C)]
 - \mathbb{E}_{Q}\left[\ln \frac{Q(S_t \mid O_t, \pi)}{Q(S_t \mid \pi)}\right]
\]

\section{Üretim Basitleştirmesi: Utility}
\[
U(a) = w_1 \cdot \mathrm{Similarity}(\mathrm{Result}(a), \mathrm{Goal})
 + w_2 \cdot \mathrm{InformationGain}(a)
 - \mathrm{Cost}(a)
\]

\newpage

% =============================================================================
% PART VII: VSA / HDC
% =============================================================================
\chapter{Hiperboyutlu Hesaplama (VSA / HDC)}

\section{Temel Operasyonlar}
İkili spatter kodları ile ($D \approx 10{,}000$):
\begin{enumerate}
  \item \textbf{Binding (XOR):} $\mathbf{u} \otimes \mathbf{v} \Rightarrow u_i \oplus v_i$
  \item \textbf{Bundling (Majority):} $\mathbf{u} + \mathbf{v} \Rightarrow [u_i + v_i > 1]$
  \item \textbf{Permutation (Shift):} $\Pi(\mathbf{u}) \Rightarrow \text{rotate}(\mathbf{u})$
\end{enumerate}

\section{Yapısal Kodlama Örneği}
\[
V_{task} = (R_{type} \otimes V_{bug}) + (R_{project} \otimes V_{fridex})
 + (R_{status} \otimes V_{pending})
\]

\newpage

% =============================================================================
% PART VIII: TESTING + AI AGENT CONFIG
% =============================================================================
\chapter{Test Süreci: AI Agent ile Sistem Analizi ve Detaylı Test Yazımı}

\section{Amaç}
Test süreci iki fazdan oluşur:
\begin{itemize}
  \item \textbf{Sistem Analizi:} değişikliğin etki alanı, riskler, performans/güvenlik etkisi.
  \item \textbf{Detaylı Test Yazımı:} kritik mantık için hedefli test; gereksiz test yok.
\end{itemize}

\section{Agent Yapılandırması (Operasyonel Sözleşme)}
FridayX üzerinde test ajanı şu kurallarla çalışır:
\begin{itemize}
  \item \textbf{Algoritmik analiz zorunlu:} her kritik fonksiyon için time/space (Big-O).
  \item \textbf{Minimal diff:} sadece gerekli satırlar.
  \item \textbf{Yasaklar:} brute-force, gereksiz nested loop, global state mutasyonu, aşırı test.
  \item \textbf{Güvenlik:} PII maskeleme, log kısıntısı, secret yönetimi.
  \item \textbf{Doğrulama:} lint + typecheck + ilgili testler; backend değiştiyse \texttt{cargo check}.
  \item \textbf{Self-review:} tek paragraf; edge-case + performans + güvenlik + karmaşıklık.
\end{itemize}

\section{Test Kapsamı Rehberi}
\begin{fridexbox}{Kritik Mantık İçin Test Örnekleri}
\begin{itemize}
  \item \textbf{LaTeX preview pipeline:} compile başarılı/başarısız durumları, log parse, debounce.
  \item \textbf{Editor split view:} state geçişleri (code/split/preview), hata gösterimi.
  \item \textbf{Performans:} büyük içerikte (örn. $> 50k$ karakter) UI kilitlenmeden derleme tetiklenmeli.
  \item \textbf{Güvenlik:} dosya sistemi kökü (filesystem root) whitelist; path traversal engeli.
\end{itemize}
\end{fridexbox}

\newpage

\chapter{Sonuç ve Vizyon Notu}
Fridex, bir araç olmanın ötesinde; insan zekasının dijital dünyadaki otonom temsilcisidir.
FridayX bu temsilin üretimde \textbf{daha şeffaf, daha güvenli ve daha hızlı} çalışması için bir işletim katmanı sağlar.

\vspace{2cm}
\begin{center}
  \textbf{Dökümanın Sonu}\\
  \textit{Bu kanonik teknik spesifikasyon Bekircan Akyüz tarafından onaylanmıştır.}
\end{center}

\end{document}
